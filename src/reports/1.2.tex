% chktex-file 44

\documentclass[a4paper,12pt]{article}
\usepackage[utf8]{inputenc}
\usepackage[russian]{babel}
\usepackage{geometry}
\usepackage{amsthm}
\usepackage[dvipsnames]{xcolor}
\usepackage{framed}
\usepackage{booktabs}
\usepackage{array}
\usepackage{amssymb}
\usepackage{adjustbox}
\usepackage{makecell}
\usepackage{float}
\usepackage{amsmath}

\definecolor{shadecolor}{RGB}{245,245,247} 
\geometry{left=2cm, right=2cm, top=2cm, bottom=2cm}

\title{Типовой расчет №1 \\ по математической статистике. \\ Часть II}
\author{Ким В.Р. \\ Группа M3207 \\ Вариант №5}
\date{}

\theoremstyle{definition}
\newtheorem{problem}{Задача}\setlength{\parindent}{0pt}

\newenvironment{solution}
{\begin{shaded}\textbf{Решение:}\par\setlength{\parindent}{0pt}}
{\end{shaded}}

\newenvironment{answer}
{\par\noindent\textbf{Ответ:}}
{\par}

\begin{document}

\maketitle

% 1
\begin{problem}
    Пассажир, приходящий в случайные моменты времени на автобусную остановку, в течение
    пяти поездок фиксировал свое время ожидания автобуса : 5,3; 3,8; 1,2; 9,2; 4,7 минуты.
    Известно, что автобус ходит с интервалом в \( \Theta \) минут. Оценить методом максимального
    правдоподобия \( \Theta \). Вычислить несмещенную оценку.

    \begin{solution}
        Описано равномерное распределение:
        \begin{equation}
            f(x) = 
            \begin{cases}
                \dfrac{1}{\Theta}, & 0 \leq x \leq \Theta), \\
                0,                 & x < 0 \lor x > \Theta).
            \end{cases}
        \end{equation}
        Выборка: \( 1.2, 3.8, 4.7, 5.3, 9.2 \), ее максимум - \(9,2\). 
        Для \(\Theta < 9.2\), \(x_{max}\) будет выходить за пределы \( [0,\Theta] \), 
        и функция распределения примет значение 0 (невозможное время ожидания автобуса)\\

        \begin{enumerate}
            \item Составим функцию правдоподобия: \\
                При \(\Theta \geq 9.2\):
                \[ L(x_1, \dots, x_5, \Theta) = \prod{i=1}^5 f(x_i, \Theta) = \prod^5_{i=1} \frac{1}{\Theta} = \frac{1}{\Theta^5}, \] тогда:

                \begin{equation}
                    L(\Theta) = 
                    \begin{cases}
                        \dfrac{1}{\Theta^5}, & \Theta \geq 9.2, \\
                        0,                  & \Theta < 9.2
                    \end{cases}
                \end{equation}
            \item  Теперь нужно максимизировать \( L(\Theta)\).\\
                   \(\dfrac{1}{\Theta^5}\) возрастает при убывании знаменателя, то есть,
                   подойдет минимальное \(\Theta\), при котором \(L(\Theta) > 0\). 
                   Единственное такое значение - это \(x_{max} = 9.2\)
        \end{enumerate}
    \end{solution}

    \begin{answer}
        9.2
    \end{answer}

\end{problem}


\vspace{8pt}
% 2
\begin{problem}
    Случайная величина \( \xi \) (число семян сорников в пробе зерна) распределена по закону Пуассона.
    Ниже прриведено распределение семян сорников в \( n = 1000 \) пробах семян. 

    \begin{table}[H]
        \centering
        \begin{adjustbox}{max width=\textwidth}
            \begin{tabular}{c c c c c c c c}
                \toprule
                \midrule
                    \(x_i\) & 0   & 1   & 2   & 3  & 4 & 5 & 6 \\
                    \(n_1\) & 405 & 366 & 175 & 40 & 8 & 4 & 2 \\
            \bottomrule
            \end{tabular}
        \end{adjustbox}
    \end{table}  

    Найти методом моментов точную оценку параметра \( \lambda \). 
    Оценить вероятность того, что в пробе зерна не будет сорняков 

    \begin{solution}
    \end{solution}

    \begin{answer}
    \end{answer}

\end{problem}


\vspace{8pt}
% 3
\begin{problem}
    Пусть \(X_1,X_2, ... ,X_n\) – случайная выборка из генеральной совокупности \(X\), 
    имеющей нормальное распределение с неизвестным средним значением \( \Theta \) 
    и известной дисперсией \( \sigma ^ 2\). Доказать, что оценка 
    \( \hat\Theta = \hat\Theta (X_1,X_2, ... ,X_n = X_1) \) является несмещенной, 
    но не является состоятельной оценкой.

    \begin{solution}
    \end{solution}

    \begin{answer}
    \end{answer}

\end{problem}


\end{document}