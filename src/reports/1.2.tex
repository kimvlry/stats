% chktex-file 44

\documentclass[a4paper,12pt]{article}
\usepackage[utf8]{inputenc}
\usepackage[russian]{babel}
\usepackage{geometry}
\usepackage{amsthm}
\usepackage[dvipsnames]{xcolor}
\usepackage{framed}
\usepackage{booktabs}
\usepackage{array}
\usepackage{amssymb}
\usepackage{adjustbox}
\usepackage{makecell}
\usepackage{float}

\definecolor{shadecolor}{RGB}{245,245,247} 
\geometry{left=2cm, right=2cm, top=2cm, bottom=2cm}

\title{Типовой расчет №1 \\ по математической статистике. \\ Часть II}
\author{Ким В.Р. \\ Группа M3207 \\ Вариант №5}
\date{}

\theoremstyle{definition}
\newtheorem{problem}{Задача}\setlength{\parindent}{0pt}

\newenvironment{solution}
{\begin{shaded}\textbf{Решение:}\par\setlength{\parindent}{0pt}}
{\end{shaded}}

\newenvironment{answer}
{\par\noindent\textbf{Ответ:} \color{blue}}
{\par}

\begin{document}

\maketitle

% 1
\begin{problem}
    Пассажир, приходящий в случайные моменты времени на автобусную остановку, в течение
    пяти поездок фиксировал свое время ожидания автобуса : 5,3; 3,8; 1,2; 9,2; 4,7 минуты.
    Известно, что автобус ходит с интервалом в \( \Theta \) минут. Оценить методом максимального
    правдоподобия \( \Theta \). Вычислить несмещенную оценку.

    \begin{solution}
    \end{solution}

    \begin{answer}
    \end{answer}

\end{problem}



% 2
\begin{problem}
    Случайная величина \( \xi \) (число семян сорников в пробе зерна) распределена по закону Пуассона.
    Ниже прриведено распределение семян сорников в \( n = 1000 \) пробах семян. 

    \begin{table}[H]
        \centering
        \begin{adjustbox}{max width=\textwidth}
            \begin{tabular}{c c c c c c c c}
                \toprule
                \midrule
                    \(x_i\) & 0   & 1   & 2   & 3  & 4 & 5 & 6 \\
                    \(n_1\) & 405 & 366 & 175 & 40 & 8 & 4 & 2 \\
            \bottomrule
            \end{tabular}
        \end{adjustbox}
    \end{table}  

    Найти методом моментов точную оценку параметра \( \lambda \). 
    Оценить вероятность того, что в пробе зерна не будет сорняков 

    \begin{solution}
    \end{solution}

    \begin{answer}
    \end{answer}

\end{problem}



% 3
\begin{problem}
    Пусть \(X_1,X_2, ... ,X_n\) – случайная выборка из генеральной совокупности \(X\), 
    имеющей нормальное распределение с неизвестным средним значением \( \Theta \) 
    и известной дисперсией \( \sigma ^ 2\). Доказать, что оценка 
    \( \hat\Theta = \hat\Theta (X_1,X_2, ... ,X_n = X_1) \) является несмещенной, 
    но не является состоятельной оценкой.

    \begin{solution}
    \end{solution}

    \begin{answer}
    \end{answer}

\end{problem}


\end{document}