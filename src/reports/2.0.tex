% chktex-file 44

\documentclass[a4paper,11pt]{article}
\usepackage[utf8]{inputenc}
\usepackage[russian]{babel}
\usepackage{geometry}
\usepackage{amsthm}
\usepackage[dvipsnames]{xcolor}
\usepackage{framed}
\usepackage{booktabs}
\usepackage{array}
\usepackage{amssymb}
\usepackage{adjustbox}
\usepackage{makecell}
\usepackage{float}
\usepackage{amsmath}
\usepackage{textcomp}
\usepackage{mdframed}
\mdfsetup{
    innerleftmargin=15pt,    
    innerrightmargin=15pt,  
    innertopmargin=10pt,    
    innerbottommargin=12pt,
    leftline=false,
    rightline=false,
    topline=false,
    bottomline=false
}
\newenvironment{shdd}{\begin{mdframed}[backgroundcolor=shadecolor]}{\end{mdframed}}

\definecolor{shadecolor}{RGB}{245,245,247} 
\geometry{left=2cm, right=2cm, top=2cm, bottom=2cm}

\title{Типовой расчет №2 \\ по математической статистике. \\ Интервальная оценка}
\author{Ким В.Р. \\ Группа M3207 \\ Вариант №5}
\date{}

\theoremstyle{definition}
\newtheorem{problem}{Задача}\setlength{\parindent}{0pt}

\newenvironment{solution}
{\begin{shdd}\textbf{Решение:}\par\setlength{\parindent}{0pt}}
{\end{shdd}}

\newenvironment{answer}
{\par\noindent\textbf{Ответ:}}
{\par}

\begin{document}

\maketitle

\begin{shdd}
    \textbf{Теоретическая справка}\par\setlength{\parindent}{0pt}
    Вероятность попадания в доверительный интервал = доверительная вероятность \(\gamma\):
    \[
    P(\hat{\theta} - \delta < \theta < \hat{\theta} + \delta) = \gamma,
    \]
    отношение \(\frac{t_{\gamma}s}{\sqrt{n}} = \gamma\) называется точностью оценки. 
    \begin{itemize}
        \item \textbf{Если \(\sigma\) (генеральное СКО) известно}:
        границы доверительного интервала определяются как
        \[
        \bar{x} +- \frac{t_{\gamma}\sigma}{\sqrt{n}},
        \]
        \(t_\gamma\) можно найти из соотношения 
        \[
        2\Phi(t_\gamma) = \gamma,
        \] 

        \item \textbf{Если \(\sigma\) неизвестно:}
        границы интервала вычисляются по похожей формуле:
        \[
        \bar{x} +- \frac{t_{\gamma}s}{\sqrt{n}},
        \]
        но вместо \(\sigma\) здесь \(s\) - его несмещенная оценка. 
        \\ А \(t_\gamma\) - значение t-распределения (распр. Стьюдента) с \(n-1\) степенями 
        свободы (табличное значение)

        \item  \textbf{Чтобы найти доверительный интервал для СКО}, используем распределение Хи-квадрат. 
        Оно связано с СКО следующим образом  
        \[
        \frac{(n-1)s^2}{\sigma^2}\sim\chi^2_{n-1},
        \]
        \newline а доверительный интервал для СКО определяется так:
        \[
        \frac{(n-1)s^2}{\chi^2_{\alpha_1, \, k}} \leq \sigma^2 \leq \frac{(n-1)s^2}{\chi^2_{\alpha_2, \, k}},
        \]
        где \(k=n-1\) - число степеней свободы, \(\alpha_1 = \frac{1-\gamma}{2}\), \(\alpha_2 = \frac{1+\gamma}{2}.\)
    \end{itemize}
\end{shdd}
\vspace{10pt}

\newpage
% 1
\begin{problem}
    Построить доверительные интервалы при значениях надежности \(\gamma=0,95\), 
    \(\gamma=0,99\), \(\gamma=0,999\) для математического ожидания случайной величины, 
    распределенной по нормальному закону. Выборка \(n=50\).
        \begin{enumerate}
            \item Каким образом величина доверительной вероятности влияет на ширину 
            доверительного интервала? 
            \item Как увеличить точность оценки при заданном значении надежности?
        \end{enumerate} 
    \begin{solution}
        \(\sigma\) неизвестно. \(t_\gamma\) находим по таблице, для \(n - 1 = 49\) степеней свободы:
        \\ \(t_{0.95} \approx 2.008\), \(t_{0.99} \approx 2.678\), \(t_{0.999} \approx 3.501\).
        \\ Интервалы: 
        \[
        \mu_{95}\in (\bar{x} - 2.008\frac{s}{\sqrt{50}};\; \bar{x} + 2.008\frac{s}{\sqrt{50}})
        \]
        \[
        \mu_{99}\in (\bar{x} - 2.678\frac{s}{\sqrt{50}};\; \bar{x} + 2.678\frac{s}{\sqrt{50}})
        \]
        \[
        \mu_{999}\in (\bar{x} - 3.501\frac{s}{\sqrt{50}};\; \bar{x} + 3.501\frac{s}{\sqrt{50}})
        \]

        \begin{enumerate}
            \item Больше \(\gamma\) - шире интервал. Формально: \(t_{\gamma}\) возрастает одновременно 
                  с \(\gamma\rightarrow\) возрастает и модуль \(\bar{x} +- \frac{t_{\gamma}\sigma}{\sqrt{n}}\).
                  Логически - чтобы с большими гарантиями покрыть истинное значение - нужно взять больший интервал.
            \item Нужно уменьшить доверительный интервал, за счет увеличения выборки 
            (больше \(n\rightarrow\) меньше \(\gamma\rightarrow\) - у`же интервал)
        \end{enumerate}
    \end{solution}

    % \begin{answer}
    % \end{answer}
\end{problem}


\vspace{10pt}
% 2
\begin{problem}
    Шестикратное взвешивание изделия из ценного материала дало следующие результаты
    (в граммах): 
    \\ 5.825, 5.844, 5.846, 5.850, 5.857, 5.861. 
    \\ Предполагая, что результаты измерений распределены по нормальному закону, требуется:
    \begin{enumerate}
        \item Найти точечные оценки математического ожидания и среднего квадратического отклонения;
        \item Найти 99\%-ный доверительный интервал, покрывающий истинный вес изделия; 
        \item Найти 95\%-ный доверительный интервал, покрывающий среднее квадратическое отклонение; 
        \item Найти предельную погрешность, которую мы допускаем, считая истинный вес изделия равным
        средней арифметической (доверительную вероятность принять равной 0,98).
    \end{enumerate} 
    \begin{solution}
        \begin{enumerate}
            \item Точечная оценка мат. ожидания - выборочное среднее: 
                \[
                \bar{x} = \frac{(5.825+5.844+5.846+5.850+5.857+5.861)}{6}\approx5.847
                \]
                
                Точечная оценка СКО - корень из исправленной выборочной дисперсии:
                \[
                s = \sqrt{\frac{1}{n-1}\sum^n_{i=1}(x_i - \bar{x})^2} 
                = \sqrt{\frac{1}{5}\sum^n_{i=1}(x_i - 5.847)^2}
                \]
                
                \[
                = \sqrt{\frac{0.022^2+2\cdot0.003^2+0.001^2+0.01^2+0.014^2}{5}} \approx 0.013
                \]

            \item 
                \(\gamma = 0.99\), то таблице: \(t_{0.99}(5) \approx 6.86\).
                \newline Границы интервала: 
                    \[ \bar{x} +- \frac{t_{\gamma}s}{\sqrt{n}}, = 5.847 +- \frac{6.86\cdot0.013}{\sqrt{6}} \rightarrow \approx (5.81; 5.88)\]
            \item Найдем доверительный интервал с помощью распределения Хи-квадрат:
                \newline \(k=n-1=5,\)
                \newline \[\alpha_1 = \frac{1-\gamma}{2} = \frac{0.05}{2} = 0.025, \quad \chi^2_{0.025, 5} = 12.8,\]
                \newline \[
                         \alpha_2 = \frac{1+\gamma}{2} = \frac{1.95}{2} = 0.975 \quad \chi^2_{0.975, 5} = 0.831
                         \]

                         \[
                         \frac{(n-1)s^2}{\chi^2_{\alpha_1, \, k}} \leq \sigma^2 \leq \frac{(n-1)s^2}{\chi^2_{\alpha_2, \, k}};
                         \quad\frac{5\cdot 0.013^2}{12.8} \leq \sigma^2 \leq \frac{5\cdot 0.013^2}{0.831};
                         \]

                         \[
                         \quad 0.0000660156 \leq\sigma^2 \leq 0.00101685;
                         \]

                         \[
                         \sqrt{0.0000660156} \leq\sigma \leq \sqrt{0.00101685}.
                         \]
        
            \item \(\gamma=0.98\)
                  \[\Delta = \frac{t_{\gamma}s}{\sqrt{n}} = \frac{4.03\cdot 0.013}{\sqrt{6}} = 0.21\]
        \end{enumerate}
    \end{solution}

    % \begin{answer}
    % \end{answer}

\end{problem}


\vspace{10pt}
% 3
\begin{problem}
    Случайная величина \(\xi\) (число сорняков в пробе зерна) распределена по закону
    Пуассона.
    Ниже приведено распределение семян сорняков в \(n\) пробах зерна \(x_i\) –
    количество сорняков в одно пробе. \(n_i\) - число проб, содержащих \(x_i\) семян сорняков.
    \\Построить доверительный интервал для параметра \(\lambda\) с надежностью \(0,95\)
    \begin{table}[H]
        \centering
        \begin{adjustbox}{max width=\textwidth}
            \begin{tabular}{c c c c c c c c c}
                \toprule
                \midrule
                    \(x_i\) & 0   & 1   & 2   & 3  & 4 & 5 & 6 & 7 \\
                    \(n_1\) & 396 & 361 & 173 & 48 & 11 & 7 & 3 & 1 \\
            \bottomrule
            \end{tabular}
        \end{adjustbox}\label{tab:table}
    \end{table}  

    
    \begin{solution}
        Закон Пуассона: 
        \[
        P(x) = \frac{\lambda^x e^{-\lambda}}{x!} 
        \]
        \(n = \sum(n_i) = 1000\). Для распределения Пуассона мат. ожидание равно \(\lambda\), 
        тогда оценку \(\hat{\lambda}\) можно найти как выборочное среднее: 
        \[
        \hat{\lambda} = \frac{361 + 173\cdot2 + 48\cdot3 + 44 + 35 + 18 + 7}{1000} = 0.955
        \] 

        Выборка большая, можем использовать асимптотическое приближение. По ЦПТ выборочное среднее 
        имеет асимптотически нормальное распределение \(N(\mu,\, \sigma^2)\), найдем его параметры:
        \begin{enumerate}
            \item \(\mu = \lambda\) по свойству распределения Пуассона
            \item Найдём дисперсию суммы \( \sum_{i=1}^n X_i \). Так как \(X_i\) независимы:
            \[
            D\left(\sum_{i=1}^n X_i\right) = \sum_{i=1}^n D(X_i) = \sum_{i=1}^n \lambda = n\lambda.
            \]
            Используем полученное значение для нахождения дисперсии оценки \(\hat{\lambda} = \frac{1}{n} \sum_{i=1}^n X_i\):
            \[
            D(\hat{\lambda}) = D\left(\frac{1}{n} \sum_{i=1}^n X_i \right) 
            = \frac{1}{n^2} \cdot D\left(\sum_{i=1}^n X_i\right) 
            = \frac{1}{n^2} \cdot n\lambda = \frac{\lambda}{n}.
            \]
            Итого: \(\hat{\lambda} \sim N\left(\lambda,\, \frac{\lambda}{n}\right).\)
        \end{enumerate}

            Теперь можно записать доверительный интервал \(\lambda\) в общем виде:
            \[
            (\hat{\lambda} - z_{\frac{3 + \gamma}{2}} \cdot \sqrt{D(\hat{\lambda})}; \; \hat{\lambda} + z_{\frac{1+\gamma}{2}} \cdot \sqrt{D(\hat{\lambda})})
            \]

            Вычислим для нашего случая: 
            Определим стандартную ошибку:
            \[
            \sqrt{D(\hat{\lambda})} = \sqrt{\frac{\hat{\lambda}}{n}} = \sqrt{\frac{0.955}{1000}} \approx 0.0309.
            \]
            
            Для доверительной вероятности \(\gamma = 0.95\), краевые значения определяются как:
            \[
            \alpha = 1 - \gamma = 0.05,\quad \frac{\alpha}{2} = 0.025.
            \]
            Тогда используем квантиль стандартного нормального распределения:
            \[
            z_{1 - \frac{\alpha}{2}} = z_{0.975} \approx 1.96.
            \]
            
            Тогда доверительный интервал для \(\lambda\) имеет вид:
            \[
            \hat{\lambda} \pm z_{0.975} \cdot SE \quad \rightarrow \quad 0.955 \pm 1.96 \cdot 0.0309.
            \]
            
            Вычисляем погрешность:
            \[
            1.96 \cdot 0.0309 \approx 0.0605.
            \]
            
            Таким образом, доверительный интервал:
            \[
            0.955 - 0.0605 \leq \lambda \leq 0.955 + 0.0605,
            \]
            
    \end{solution}

    \begin{answer}
        (0.8945;\, 1.0155)
    \end{answer}

\end{problem}


\vspace{10pt}
% 4
\begin{problem}
    Для экспоненциального распределения “со сдвигом” , имеющего плотность
    \begin{equation}
        f(x; \theta) = 
        \begin{cases}
            e^{-x-\theta}, & x\geq\theta \\
            0, &x<\theta
        \end{cases}\label{eq:equation}
    \end{equation}
    
    по выборке объема \(n\) построить интервальную оценку параметра \(\theta\) (сдвиг) с доверительной
    вероятностью \(\gamma\)
    \begin{solution}
        Сделав замену переменной \(y=x-\theta\), получаем, что \(y\ge0\) и функция плотности
        \[
        f_Y(y)= e^{-y},\quad y \ge 0,
        \]
        это плотность стандартного экспоненциального распределения \( \operatorname{Exp}(1), \) \( \lambda = 1 \).
    
        Пусть \(X_1, X_2, \dots, X_n\) — независимая выборка из данного распределения. Заметим, что условие \(x \ge \theta\) 
        для всех наблюдений позволяет нам использовать минимальное значение выборки как статистику, содержащую всю информацию 
        о сдвиге \(\theta\). Обозначим
        \[
        X_{min} = \min\{X_1, X_2, \dots, X_n\}.
        \]
    
        Каждое наблюдение \(X_i\) можно представить в виде
        \[
        X_i = \theta + Y_i,\quad \text{где } Y_i\sim \operatorname{Exp}(1),
        \]
        где $Y_1, Y_2, \dots, Y_n$ — независимые случайные величины.
        \[
        Y_i \sim \mathrm{Exp}(1), \quad \text{с плотностью } f_{Y_i}(y) = e^{-y}, \ y \geq 0
        \]
        Обозначим \( Z = \min\{Y_1, Y_2, \dots, Y_n\} \)\
        Тогда минимум выборки - \( X_{min} = \theta + Z, Z = X_{min} - \theta\)

        \subsubsection*{Функция распределения минимума}
        Рассмотрим $Z = \min\{Y_1, Y_2, \dots, Y_n\}$.

        Найдём её функцию распределения $F_Z(z)$.
        \begin{align*}
        F_Z(z) &= P(Z \leq z) = 1 - P(Z > z) \\
        &= 1 - P(\min\{Y_1, \dots, Y_n\} > z) \\
        &= 1 - P(Y_1 > z \cap \dots \cap Y_n > z) \\
        &= 1 - \prod_{i=1}^n P(Y_i > z) \quad \text{(по независимости)} \\
        &= 1 - \prod_{i=1}^n e^{-z} \quad \text{(т.к. } P(Y_i > z) = e^{-z}) \\
        &= 1 - e^{-nz}, \quad z \geq 0
        \end{align*}

%        \subsubsection*{Плотность распределения минимума}
%        Дифференцируя функцию распределения, получаем плотность:
%        \[
%        f_Z(z) = \frac{d}{dz}F_Z(z) = n e^{-nz}, \quad z \geq 0
%        \]
%        Это соответствует плотности экспоненциального распределения с параметром $n$:
%        \[
%        Z \sim \mathrm{Exp}(n)
%        \]

        Мы ищем такое множество \(I\), что
        \[
        P(\theta \in I) = \gamma.
        \]
        
        Используя распределение \(Z\), найдем квантиль \(z_\gamma\) такой, что \(P(Z \le z_\gamma) = \gamma\).
        \\Имея функцию распределения \(F_Z(z)=1-e^{-nz}\) получаем:
        \[
        1-e^{-nz_\gamma}=\gamma \quad \Rightarrow \quad e^{-nz_\gamma}=1-\gamma,
        \]
        выразим \(z_{\gamma}\)
        \[
        z_\gamma=-\frac{1}{n}\ln(1-\gamma).
        \]
    
        Подставим \(Z=X_{min}-\theta\):
        \[
        P\Bigl(X_{min} -\theta \le z_\gamma \Bigr)=\gamma \Leftrightarrow P\Bigl(\theta \ge X_{min} - z_\gamma \Bigr)=\gamma.
        \]
        Так как сверху \(\theta\) ограничена \(X_{min}\), получаем
        \(\theta \in \Bigl(X_{min} - z_\gamma,\, X_{min}\Bigr)\).

        Подставляя \(z_{\gamma}\):
        \[
        P\left(X_{min} - \frac{1}{n}\ln\frac{1}{1-\gamma} \le \theta \le X_{min}\right)=\gamma.
        \]
        (перепишем \(-\frac{1}{n}\ln(1-\gamma)=\frac{1}{n}\ln\left(\frac{1}{1-\gamma}\right)\))

        \vspace{10pt}
        Итого, интервал для параметра \(\theta\) с доверительной вероятностью \(\gamma\) имеет вид:
        \[
        \theta \in \left(X_{min} - \frac{1}{n}\ln\left(\frac{1}{1-\gamma}\right),\, X_{min}\right).
        \]
    \end{solution}
    

    % \begin{answer}
    % \end{answer}

\end{problem}

\vspace{25pt}

\end{document}
