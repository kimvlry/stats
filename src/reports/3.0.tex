% chktex-file 44

\documentclass[a4paper,11pt]{article}
\usepackage[utf8]{inputenc}
\usepackage[russian]{babel}
\usepackage{geometry}
\usepackage{amsthm}
\usepackage[dvipsnames]{xcolor}
\usepackage{framed}
\usepackage{booktabs}
\usepackage{array}
\usepackage{amssymb}
\usepackage{adjustbox}
\usepackage{makecell}
\usepackage{float}
\usepackage{amsmath}
\usepackage{textcomp}
\usepackage{mdframed}
\mdfsetup{
    innerleftmargin=15pt,
    innerrightmargin=15pt,
    innertopmargin=10pt,
    innerbottommargin=12pt,
    leftline=false,
    rightline=false,
    topline=false,
    bottomline=false
}
\newenvironment{shdd}{\begin{mdframed}[backgroundcolor=shadecolor]}{\end{mdframed}}

\definecolor{shadecolor}{RGB}{245,245,247}
\geometry{left=2cm, right=2cm, top=2cm, bottom=2cm}

\title{Типовой расчет №3 \\ по математической статистике. \\ Статистические гипотезы}
\author{Ким В.Р. \\ Группа M3207 \\ Вариант №5}
\date{}

\theoremstyle{definition}
\newtheorem{problem}{Задача}\setlength{\parindent}{0pt}

\newenvironment{solution}
{\begin{shdd}
     \textbf{Решение:}\par\setlength{\parindent}{0pt}}
     {
\end{shdd}}

\newenvironment{answer}
{\par\noindent\textbf{Ответ:}}
{\par}



\begin{document}

    \maketitle

    \begin{shdd}
        \textbf{Теоретическая справка}\par\setlength{\parindent}{0pt}
        \begin{itemize}
            \item Нулевая гипотеза \(H_0 : \theta = \theta_0\)
            \item Альтернативная гипотеза \(H_1\)
            \begin{itemize}
                \item двусторонняя: \(\theta \neq \theta_0\)
                \item односторонняя \(\theta > \theta_0, \theta < \theta_0\)
            \end{itemize}
            \item Ошибка I рода - отклонение верной \(H_0\)
            \begin{itemize}
                \item \(P(I) = \alpha\) - уровень значимости
                \item Ошибка II рода
            \end{itemize}
            \item Ошибка II рода - принятие неверной \(H_1\)
            \begin{itemize}
                \item \(P(II) = \beta\)
                \item \(1 - \beta\) - мощность критерия
                или вероятность правильно отвергнуть \(H_0\), когда она ложна.
            \end{itemize}
        \end{itemize}
    \end{shdd}
    \vspace{10pt}



    \newpage
% 1
    \begin{problem}
        По выборке объема \(n = 36\), извлеченной из нормальной генеральной совокупности
        с известным средним квадратическим отклонением \(\sigma = 6\), на уровне значимости
        \(\alpha = 0,01\) проверяется нулевая гипотеза \(H_0\):
        \(a = a_0 = 15\) при конкурирующей гипотезе \(H_1 : a = a_0 \neq 15\).\\
        \\
        Найти мощность \((1-\beta)\) двустороннего критерия проверки рассматриваемой гипотезы для \(a_1 = 12\).

        \begin{solution}
%            Найдем границы допустимой области:
%            \begin{gather*}
%                \Phi(k_{crit}) = \frac{1 - \alpha}{2} = \frac{0,995}{2} = 0,4975 \Longrightarrow k_{crit} = 2,08\\
%                \Phi(-k_{crit}) = -\Phi(k_{crit}) \Longrightarrow -k_{crit} = -2,08\\
%            \end{gather*}
%            итого, допустимая область: \([-2,08; 2,08]\)
%
%            \[
%                U = \frac{\bar{x} - a_0}{\frac{\sigma}{\sqrt {n}}} = \frac{\bar{x} - 15}{6/\sqrt{36}} = \bar{x} - 15
%            \]
%
%            Найдем мощность рассматриваемого критерия, что по определению есть вероятность
%            попадания статистики критерия в критическую область при допущении, что справедлива
%            конкурирующая гипотеза
%
%            \begin{gather*}
%                1 - \beta = P\left(\left| \frac{\bar{x} - a_0}{\frac{\sigma}{\sqrt{n}}} \right| > k_{\text{crit}} \,\middle|\, H_1\right); \\
%                \left| \frac{\bar{x} - a_0}{\frac{\sigma}{\sqrt{n}}} \right| > k_{\text{crit}} \\
%                \Leftrightarrow \\
%                \bar{x} - a_0 < -k_{\text{crit}} \;\cup\; \bar{x} - a_0 > k_{\text{crit}} \\
%                \Leftrightarrow \\
%                \bar{x} < a_0 - k_{\text{crit}} \;\cup\; \bar{x} > a_0 + k_{\text{crit}} \\
%                \bar{x} < 12,92 \;\cup\; \bar{x} > 17,08
%            \end{gather*}
%            если \(\bar{x}\) оказывается в этих пределах, то \(H_0\) отвергается.
%            \\
%            Теперь найдем мощность

        \end{solution}

        \begin{answer}
        \end{answer}

    \end{problem}



    \vspace{10pt}
% 2
    \begin{problem}
        Из нормальной генеральной совокупности извлечена выборка объема \(n=17\) и по ней
        найдена исправленная выборочная дисперсия \(s^2 = 0,24\).
        Требуется при уровне значимости \(a = 0,05\) проверить нулевую гипотезу \(H_0: \sigma^2 = 18\),
        приняв в качестве альтернативной гипотезы \(H_1: \sigma^2 > 0,18\)

        \begin{solution}
        \end{solution}


        \begin{answer}
        \end{answer}


    \end{problem}



    \vspace{10pt}
% 3
    \begin{problem}
        По группировке, полученной в Практической работе №1(часть1), используя критерий \(\chi^2\),
        проверить при уровнях значимости \(0,05\) и \(0,01\) гипотезу о нормальном распределении
        соответствующего признака взяв в качестве значений параметров нормального распределения
        их оценки, полученные по сгруппированным данным

        \begin{solution}
        \end{solution}

        \begin{answer}
        \end{answer}

    \end{problem}



    \vspace{10pt}
% 4
    \begin{problem}
        По выборке объема \(n = 30\) найден средний вес \(x = 130\)г изделий, изготовленных на
        первом станке; по выборке объема \(m = 40\) найден средний вес \(\bar{y} = 125\)г изделий,
        изготовленных на втором станке.
        Генеральные дисперсии известны: \(D(X) = 60\)г, \(D(Y) = 80\)г.
        Требуется при уровне значимости \(0,05\) проверить нулевую гипотезу \(H_0 : M(X) = M(Y)\)
        при конкурирующей гипотезе \(H_1 : M(X) \neq M(Y)\).
        Предполагается, что случайные величины \(X\)и\(Y\) распределены нормально и выборки независимы

        \begin{solution}
        \end{solution}

        \begin{answer}
        \end{answer}

    \end{problem}


    \vspace{10pt}
% 5
    \begin{problem}
        В результате взвешивания \(800\) стальных шариков получено эмпирическое распределение,
        приведенное в таблице ( в первом столбце указан интервал веса в граммах, во втором –
        частота, то есть количество шариков, вес которых принадлежит этому интервалу.
        Требуется при уровне значимости \(0,01\) проверить гипотезу о том,
        что вес шариков \(X\) распределен равномерно

        \begin{table}[H]
    \centering
    \begin{adjustbox}{max width=\textwidth}
        \begin{tabular}{l c c}
            \toprule
            \makecell{\(X_{i-1} - X_i\)} & \makecell{\(n_i\)} & \\
            \midrule
            20.0-20.5 & 91 \\
            20.5-21.0 & 76 \\
            21.0-21.5 & 75 \\
            21.5-22.0 & 74 \\
            22.0-22.5 & 92 \\
            22.5-23.0 & 83 \\
            23.0-23.5 & 79 \\
            23.5-24.0 & 73 \\
            24.0-24.5 & 80 \\
            24.5-25.0 & 77 \\
            \bottomrule
        \end{tabular}
    \end{adjustbox}
    \label{tab:table}
\end{table}


        \begin{solution}
        \end{solution}

        \begin{answer}
        \end{answer}

    \end{problem}


\end{document}
