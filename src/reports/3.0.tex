% chktex-file 44

\documentclass[a4paper,11pt]{article}
\usepackage[utf8]{inputenc}
\usepackage[russian]{babel}
\usepackage{geometry}
\usepackage{amsthm}
\usepackage[dvipsnames]{xcolor}
\usepackage{framed}
\usepackage{booktabs}
\usepackage{array}
\usepackage{amssymb}
\usepackage{adjustbox}
\usepackage{makecell}
\usepackage{float}
\usepackage{amsmath}
\usepackage{textcomp}
\usepackage{mdframed}
\mdfsetup{
    innerleftmargin=15pt,
    innerrightmargin=15pt,
    innertopmargin=10pt,
    innerbottommargin=12pt,
    leftline=false,
    rightline=false,
    topline=false,
    bottomline=false
}
\newenvironment{shdd}{\begin{mdframed}[backgroundcolor=shadecolor]}{\end{mdframed}}

\definecolor{shadecolor}{RGB}{245,245,247}
\geometry{left=2cm, right=2cm, top=2cm, bottom=2cm}

\title{Типовой расчет №3 \\ по математической статистике. \\ Статистические гипотезы}
\author{Ким В.Р. \\ Группа M3207 \\ Вариант №5}
\date{}

\theoremstyle{definition}
\newtheorem{problem}{Задача}\setlength{\parindent}{0pt}

\newenvironment{solution}
{\begin{shdd}
     \textbf{Решение:}\par\setlength{\parindent}{0pt}}
     {
\end{shdd}}

\newenvironment{answer}
{\par\noindent\textbf{Ответ:}}
{\par}



\begin{document}

    \maketitle

    \begin{shdd}
        \textbf{Теоретическая справка}\par\setlength{\parindent}{0pt}
        \begin{itemize}
            \item Нулевая гипотеза \(H_0 : \theta = \theta_0\)
            \item Альтернативная гипотеза \(H_1\)
            \begin{itemize}
                \item двусторонняя: \(\theta \neq \theta_0\)
                \item односторонняя \(\theta > \theta_0, \theta < \theta_0\)
            \end{itemize}
            \item Ошибка I рода - отклонение верной \(H_0\)
            \begin{itemize}
                \item \(P(I) = \alpha\) - уровень значимости
                \item Ошибка II рода
            \end{itemize}
            \item Ошибка II рода - принятие неверной \(H_1\)
            \begin{itemize}
                \item \(P(II) = \beta\)
                \item \(1 - \beta\) - мощность критерия
                или вероятность правильно отвергнуть \(H_0\), когда она ложна.
            \end{itemize}
            \item Проверка гипотезы о виде распределения
            \begin{itemize}
                \item требует вычисления теоретических частот \(n_i^{'}\) (согласно предполагаемому распределению)
                \item проверяется в помощью критерия Пирсона: \(\chi^2_{\text{exp}} = \sum_{i=1}^k \frac{(n_i - n_i')^2}{n_i'}\)
                \item если \(\chi^2_{\text{exp}} < \chi^2_{\text{crit}}(\alpha; k) \), гипотеза принимается
            \end{itemize}
            \item Гипотеза о равенстве мат. ожиданий
            \begin{itemize}
                \item если генеральные дисперсии известны, критерий: \(z = \frac{\bar{x} - \bar{y}}{\sqrt{\frac{D(X)}{n} + \frac{D(Y)}{m}}}\)
                \item если не известны, но считаются равными:
                \begin{itemize}
                    \item используем критерий \(t = \frac{\bar{x} - \bar{y}}{D_p \sqrt{\frac{1}{n_1} + \frac{1}{n_2}}}\),
                    где \(\bar{x}\) и \(\bar{y}\) — выборочные средние, \(n_1\) и \(n_2\) — объемы выборок, \(D_p\) — объединенная оценка стандартного отклонения
                    \item \(D_p = \sqrt{\frac{(n_1 - 1)D(X) + (n_2 - 1)D(Y)}{n_1 + n_2 - 2}}\), где \(D(X)\) и \(D(Y)\) — выборочные дисперсии.
                    \item число степеней свободны t-распределения: \(k = n_1 + n_2 - 2\)
                \end{itemize}
            \end{itemize}
        \end{itemize}
    \end{shdd}
    \vspace{10pt}



    \newpage
% 1
    \begin{problem}
        По выборке объема \(n = 36\), извлеченной из нормальной генеральной совокупности
        с известным средним квадратическим отклонением \(\sigma = 6\), на уровне значимости
        \(\alpha = 0,01\) проверяется нулевая гипотеза \(H_0\):
        \(a = a_0 = 15\) при конкурирующей гипотезе \(H_1 : a = a_0 \neq 15\).
        \\
        \\
        Найти мощность \((1-\beta)\) двустороннего критерия проверки рассматриваемой гипотезы для \(a_1 = 12\).

        \begin{solution}
            Найдем границы допустимой области:
            \begin{gather*}
                \Phi(k_{crit}) = \frac{1 - \alpha}{2} = \frac{0,995}{2} = 0,4975 \Longrightarrow k_{crit} = 2,08\\
                \Phi(-k_{crit}) = -\Phi(k_{crit}) \Longrightarrow -k_{crit} = -2,08\\
            \end{gather*}
            итого, допустимая область: \([-2,08; 2,08]\)

            \[
                U = \frac{\bar{x} - a_0}{\frac{\sigma}{\sqrt {n}}} = \frac{\bar{x} - 15}{6/\sqrt{36}} = \bar{x} - 15
            \]

            Найдем мощность рассматриваемого критерия, что по определению есть вероятность
            попадания статистики критерия в критическую область при допущении, что справедлива
            конкурирующая гипотеза

            \begin{gather*}
                1 - \beta = P\left(\left| \frac{\bar{x} - a_0}{\frac{\sigma}{\sqrt{n}}} \right| > k_{\text{crit}} \,\middle|\, H_1\right); \\
                \left| \frac{\bar{x} - a_0}{\frac{\sigma}{\sqrt{n}}} \right| > k_{\text{crit}} \\
                \Leftrightarrow \\
                \bar{x} - a_0 < -k_{\text{crit}} \;\cup\; \bar{x} - a_0 > k_{\text{crit}} \\
                \Leftrightarrow \\
                \bar{x} < a_0 - k_{\text{crit}} \;\cup\; \bar{x} > a_0 + k_{\text{crit}} \\
                \bar{x} < 12,92 \;\cup\; \bar{x} > 17,08
            \end{gather*}
            если \(\bar{x}\) оказывается в этих пределах, то \(H_0\) отвергается.\\

            Если истинное значение \(a = 12\), то:
            \[
                \bar{x} \sim N\left(a_1, \frac{\sigma^2}{n}\right) = N\left(12, \frac{6^2}{36}\right) = N(12, 1).
            \]

            Преобразуем критические значения \(\bar{x}\) в \(Z\)-оценки:
            \[
                Z_{\text{лев}} = \frac{12.42 - 12}{1} = 0.42, \quad Z_{\text{прав}} = \frac{17.58 - 12}{1} = 5.58.
            \]

            Расчет мощности критерия:
            \\
            Мощность \((1-\beta)\) — это вероятность попадания \(\bar{x}\) в критическую область при \(a = 12\):
            \[
                1 - \beta = P(\bar{x} < 12.42 \mid a=12) + P(\bar{x} > 17.58 \mid a=12).
            \]

            \begin{itemize}
                \item Левая критическая область:
                \[
                    P(\bar{x} < 12.42 \mid a=12) = P\left(Z < 0.42\right) \approx 0.6628 \quad (\text{из таблиц } N(0,1)).
                \]

                \item Правая критическая область:
                \[
                    P(\bar{x} > 17.58 \mid a=12) = P\left(Z > 5.58\right) \approx 1 - \Phi(5.58) \approx 1 - 0.999999 = 0.000001.
                \]
            \end{itemize}

            Итоговая мощность:
            \[
                1 - \beta \approx 0.6628 + 0.000001 \approx 0.66
            \]
        \end{solution}

        \begin{answer}
            0,66
        \end{answer}

    \end{problem}



    \vspace{10pt}
% 2
    \begin{problem}
        Из нормальной генеральной совокупности извлечена выборка объема \(n=17\) и по ней
        найдена исправленная выборочная дисперсия \(s^2 = 0,24\).
        \\
    
        Требуется при уровне значимости \(a = 0,05\) проверить нулевую гипотезу \(H_0: \sigma_0^2 = 18\),
        приняв в качестве альтернативной гипотезы \(H_1: \sigma_0^2 > 0,18\)

        \begin{solution}
            Так как дана исправленная выборочная дисперсия, для определения статистики критерия
            и границы критической области будем использовать распределение Хи-квадрат.
            \\
        
            Исходя из вида \(H_1\), критическая область правосторонняя.
            Найдем критическую границу. У нас \(n - 1 = 16\) степеней свободы и \( \alpha = 1 - a = 0,95 \):
            \[
                \chi^2_{\text{crit}} = \chi^2_{0,95}(16) = 7,962
            \]
            Посчитаем статистику критерия
            \[
                U = \frac{(n-1)\cdot s^2}{\sigma_0^2} = \frac{16 \cdot 0,24}{18} \approx 0,213,
            \]
            так как \(0,213 < 7,962\), гипотеза \(H_0\) принимается.
        \end{solution}

        \begin{answer}
            \(H_0\) принимается
        \end{answer}


    \end{problem}



    \vspace{10pt}
% 3
    \begin{problem}
        По группировке, полученной в типовом расчете №1 (часть1), используя критерий \(\chi^2\),
        проверить при уровнях значимости \(0,05\) и \(0,01\) гипотезу о нормальном распределении
        соответствующего признака взяв в качестве значений параметров нормального распределения
        их оценки, полученные по сгруппированным данным

        \begin{solution}
            Из первого типовика:
            \begin{table}[H]
                \centering
                \begin{adjustbox}{max width=\textwidth}
                    \begin{tabular}{l c c c c c c c}
                        \toprule
                        \makecell{Номер \\интервала \( i \)} &
                        \makecell{Границы \\ интервала \( i \)} &
                        \makecell{Середина \\интервала \( x_i \)} &
                        \makecell{Частота \\\( n_i \)} &
                        \makecell{Отн. частота \\\( W_i = \frac{n_i}{N} \)} \\
                        \midrule
                        1 & [5.13; 7.24)   & 6.18  & 1  & 0.01    \\
                        2 & [7.24; 9.35)  & 8.29 & 0  & 0 \\
                        3 & [9.35; 11.46) & 10.40 & 2 & 0.02 \\
                        4 & [11.46; 13.56) & 12.51 & 19 & 0.19 \\
                        5 & [13.56; 15.67) & 14.62 & 46 & 0.46 \\
                        6 & [15.67; 17.78) & 16.73 & 28  & 0.28 \\
                        7 & [17.78; 19.89) & 18.84 & 4  & 0.04 \\
                        \bottomrule
                    \end{tabular}
                \end{adjustbox}\label{tab:table2}
            \end{table}

            Выборочное среднее \(\bar{x} = 14.81\), несмещенная оценка дисперсии \(\hat{s^2} = 3.88 \),
            среднеквадратичное отклонение = \( s = 1.97 \).
            \\
            
            Для нормального распределения вероятность попадания в интервал выглядит так:
            \[
                P(\alpha < x < \beta) = \Phi(\frac{\beta - a}{\sigma}) - \Phi(\frac{\alpha - a}{\sigma}).
            \]
            Посчитаем теоретические частоты.
            \begin{enumerate}
                \item \(\Phi\left(\frac{7.24 - 14.81}{1.97}\right) - \Phi\left(\frac{5.13 - 14.81}{1.97}\right) = \Phi(-3.84) - \Phi(-4.91) \approx 0.00008 \), \\[1em]
                \(n_i^{'} = 100 \cdot 0.00008 = 0.008\)

                \item \(\Phi\left(\frac{9.35 - 14.81}{1.97}\right) - \Phi\left(\frac{7.24 - 14.81}{1.97}\right) = \Phi(-2.77) - \Phi(-3.84) \approx 0.499992 - 0.4971 \approx 0.003\), \\[1em]
                \(n_i^{'} = 100 \cdot 0.003 = 0.3\)

                \item \(\Phi\left(\frac{11.46 - 14.81}{1.97}\right) - \Phi\left(\frac{9.35 - 14.81}{1.97}\right) = \Phi(-1.7) - \Phi(-2.77) = 0.4973 - 0.4554 \approx 0.04\), \\[1em]
                \(n_i^{'} = 4\)

                \item \(\Phi\left(\frac{13.56 - 14.81}{1.97}\right) - \Phi\left(\frac{11.46 - 14.81}{1.97}\right) = \Phi(-0.63) - \Phi(-1.7) = 0.4554 - 0.2357 \approx 0.22\), \\[1em]
                \(n_i^{'} = 22\)

                \item \(\Phi\left(\frac{15.67 - 14.81}{1.97}\right) - \Phi\left(\frac{13.56 - 14.81}{1.97}\right) = \Phi(0.43) - \Phi(-0.63) = 0.2357 + 0.1664 \approx 0.4\), \\[1em]
                \(n_i^{'} = 40\)

                \item \(\Phi\left(\frac{17.78 - 14.81}{1.97}\right) - \Phi\left(\frac{15.67 - 14.81}{1.97}\right) = \Phi(1.5) - \Phi(0.43) = 0.4332 - 0.1664 \approx 0.27\), \\[1em]
                \(n_i^{'} = 27\)

                \item \(\Phi\left(\frac{19.89 - 14.81}{1.97}\right) - \Phi\left(\frac{17.78 - 14.81}{1.97}\right) = \Phi(2.57) - \Phi(1.5) = 0.495 - 0.4332 \approx 0.062\), \\[1em]
                \(n_i^{'} = 6.2 \)
            \end{enumerate}
            На первых 3 интервалах n_i^{'} слишком мало (\( < 5\)), объединим их чтобы не получить большую погрешность
            при расчете \(\chi^2_{\text{exp}}\). \(n_{1-3}^{'} = 0.008 + 0.3 + 4 = 4.308\).

            Сравним экспериментальные частоты из 1 типовика и полученные теоретические частоты, для проверки.
            \begin{table}[H]
                \centering
                \begin{adjustbox}{max width=\textwidth}
                    \begin{tabular}{l c c c}
                        \toprule
                        \makecell{ \( i \)} &
                        \makecell{ \( n_i \)} &
                        \makecell{ \( n_i^{'} \)} \\
                        \midrule
                        1..3 & 3  & 4.308 \\
                        4    & 19 & 22    \\
                        5    & 46 & 40    \\
                        6    & 28 & 27    \\
                        7    & 4  & 6.2   \\
                        \bottomrule
                    \end{tabular}
                \end{adjustbox}\label{tab:table3}
            \end{table}
            ну вроде похоже.

            \newpage
            Критерий согласия Пирсона:
            \[
                \chi^2_{\text{exp}} = \sum\frac{(n_i - n^{'}_i)^2}{n^{'}_i}
            \]
            \begin{gather*}
                \chi^2_{\text{exp}} =
                    {\frac{(3 - 4.308)^2}{4.308}} +
                    {\frac{(19 - 22)^2}{22}} +
                    {\frac{(46 - 40)^2}{40}} +
                    {\frac{(28 - 27)^2}{27}} +
                    {\frac{(4 - 6.2)^2}{6.2}} \approx 2.52
            \end{gather*}

            Число степеней свободы для Хи-квадлрат распределения определяется так: \(k = n - p - 1\),
            у нас: \(k = 7 - 2 - 1 = 4\) (\(p=2\) - два параметра нормального распределения).\\
            Критическое \( \chi^2_{\text{crit}}(\alpha; k) \) находится из таблицы:
            \begin{itemize}
                \item \( \chi^2_{\text{crit}}(0,05; 4) = 9.5\)
                \item \( \chi^2_{\text{crit}}(0,01; 4) = 13.3\)
            \end{itemize}
            Если \( \chi^2_{\text{exp}} < \( \chi^2_{\text{crit}}(\alpha; k) \), гипотеза принимается.
            В нашем случае: \( \chi^2_{\text{exp}} < \( \chi^2_{\text{crit}}(\alpha; k) \) при обоих \(\alpha\)
        \end{solution}

        \begin{answer}
            гипотеза принимается в обоих случаях
        \end{answer}

    \end{problem}



    \vspace{10pt}
% 4
    \begin{problem}
        По выборке объема \(n = 30\) найден средний вес \(x = 130\)г изделий, изготовленных на
        первом станке; по выборке объема \(m = 40\) найден средний вес \(\bar{y} = 125\)г изделий,
        изготовленных на втором станке.
        Генеральные дисперсии известны: \(D(X) = 60\)г, \(D(Y) = 80\)г.
        \\
        
        Требуется при уровне значимости \(0,05\) проверить нулевую гипотезу \(H_0 : M(X) = M(Y)\)
        при конкурирующей гипотезе \(H_1 : M(X) \neq M(Y)\).
        Предполагается, что случайные величины \(X\)и\(Y\) распределены нормально и выборки независимы

        \begin{solution}
            Критерий для проверки гипотезы о равенстве
            мат. ожиданий при известных генеральных дисперсиях выглядит вот так:
            \[
                z = \frac{\bar{x} - \bar{y}}{\sqrt{\frac{D(X)}{n} + \frac{D(Y)}{m}}}.
            \]

            В нашем случае:
            \begin{gather*}
                \bar{x} - \bar{y} = 130 - 125 = 5.\\
                \sqrt{\frac{D(X)}{n} + \frac{D(Y)}{m}} = \sqrt{\frac{60}{30} + \frac{80}{40}} = \sqrt{2 + 2} = \sqrt{4} = 2.\\
                z = \frac{\bar{x} - \bar{y}}{\sqrt{\frac{D(X)}{n} + \frac{D(Y)}{m}}} = \frac{5}{2} = 2.5.\\
            \end{gather*}

            Гипотеза двусторонняя (\(H_1: M(X) \neq M(Y)\)), поэтому уровень значимости \(\alpha = 0.05\) делим пополам: \(\alpha/2 = 0.025\).
            Критическое значение \(z_{\text{crit}}\) для стандартного нормального распределения:
            \[
                P(Z > z_{\text{crit}}) = 0.025 \implies z_{\text{crit}} = 1.96.
            \]
            Критическая область: \(|z| > 1.96\).

            \(|z| = 2.5 > 1.96\), значение \(z = 2.5\) попадает в критическую область.

            На уровне значимости \(\alpha = 0.05\) отвергаем нулевую гипотезу \(H_0: M(X) = M(Y)\), так как \(|z| = 2.5 > 1.96\).
            Средние веса изделий на первом и втором станках различаются.
        \end{solution}

        \begin{answer}
            гипотеза отвергается в пользу альтернативной
        \end{answer}

    \end{problem}


    \vspace{10pt}
% 5
    \begin{problem}
        В результате взвешивания \(800\) стальных шариков получено эмпирическое распределение,
        приведенное в таблице (в первом столбце указан интервал веса в граммах, во втором –
        частота, то есть количество шариков, вес которых принадлежит этому интервалу).
        \\
        
        Требуется при уровне значимости \(0,01\) проверить гипотезу о том,
        что вес шариков \(X\) распределен равномерно

        \begin{table}[H]
            \centering
            \begin{adjustbox}{max width=\textwidth}
                \begin{tabular}{l c c}
                    \toprule
                    \makecell{\(X_{i-1} - X_i\)} & \makecell{\(n_i\)} & \\
                    \midrule
                    20.0-20.5 & 91 \\
                    20.5-21.0 & 76 \\
                    21.0-21.5 & 75 \\
                    21.5-22.0 & 74 \\
                    22.0-22.5 & 92 \\
                    22.5-23.0 & 83 \\
                    23.0-23.5 & 79 \\
                    23.5-24.0 & 73 \\
                    24.0-24.5 & 80 \\
                    24.5-25.0 & 77 \\
                    \bottomrule
                \end{tabular}
            \end{adjustbox}
            \label{tab:table}
        \end{table}


        \begin{solution}
            Для проверки гипотезы о виде распределения используем критерий \(\chi^2\).

            Равномерное распределение на отрезке \([a, b]\) имеет плотность \(f(x) = \frac{1}{b - a}\).
            Из таблицы: \(a = 20\), \(b = 25\), тогда:
            \[
                b - a = 25 - 20 = 5, \quad f(x) = \frac{1}{5} = 0.2.
            \]

            Всего \(k = 10\) интервалов, каждый длиной \(0.5\) г. Вероятность попадания в каждый интервал:
            \[
                P(X \in [x_{i-1}, x_i)) = f(x) \cdot (x_i - x_{i-1}) = 0.2 \cdot 0.5 = 0.1.
            \]
            Теоретические частоты очевидно будут одинаковы для всех интервалов
            и равны числу \(n_i' = n \cdot P = 800 \cdot 0.1 = 80\). Все \(n_i' \geq 5\), объединение интервалов не требуется.\\
            \\
            Вычислим статистику для критерия согласия Пирсона \(\chi^2_{\text{exp}} = \sum_{i=1}^k \frac{(n_i - n_i')^2}{n_i'}\):
            \begin{itemize}
                \item \(\frac{(91 - 80)^2}{80} = \frac{11^2}{80} = \frac{121}{80} = 1.5125\)\\
                \item \(\frac{(76 - 80)^2}{80} = \frac{(-4)^2}{80} = \frac{16}{80} = 0.2\)\\
                \item \(\frac{(75 - 80)^2}{80} = \frac{(-5)^2}{80} = \frac{25}{80} = 0.3125\)\\
                \item \(\frac{(74 - 80)^2}{80} = \frac{(-6)^2}{80} = \frac{36}{80} = 0.45\)\\
                \item \(\frac{(92 - 80)^2}{80} = \frac{12^2}{80} = \frac{144}{80} = 1.8\)\\
                \item \(\frac{(83 - 80)^2}{80} = \frac{3^2}{80} = \frac{9}{80} = 0.1125\)\\
                \item \(\frac{(79 - 80)^2}{80} = \frac{(-1)^2}{80} = \frac{1}{80} = 0.0125\)\\
                \item \(\frac{(73 - 80)^2}{80} = \frac{(-7)^2}{80} = \frac{49}{80} = 0.6125\)\\
                \item \(\frac{(80 - 80)^2}{80} = 0\)\\
                \item \(\frac{(77 - 80)^2}{80} = \frac{(-3)^2}{80} = \frac{9}{80} = 0.1125\)\\
            \end{itemize}
            \[
                \chi^2_{\text{exp}} = 1.5125 + 0.2 + 0.3125 + 0.45 + 1.8 + 0.1125 + 0.0125 + 0.6125 + 0 + 0.1125 = 5.125.
            \]

            Число степеней свободы: \(k = n - p - 1\), где \(k = 10\), \(p\) — число оцениваемых параметров.
            Для равномерного распределения параметры \(a\) и \(b\) известны, тогда (\(p = 0\)):
            \[
                k = 10 - 0 - 1 = 9.
            \]

            Заглянем в таблицу Хи-квадрат распределения
            \[
                \chi^2_{\text{crit}}(0.01; 9) = 21.7.
            \]

            Сравниваем: \(\chi^2_{\text{exp}} = 5.125 < \chi^2_{\text{crit}} = 21.7\), то есть \(H_0\) принимается

        \end{solution}

        \begin{answer}
            гипотеза принимается
        \end{answer}

    \end{problem}


\end{document}
