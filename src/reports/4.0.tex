% chktex-file 44

\documentclass[a4paper,11pt]{article}
\usepackage[utf8]{inputenc}
\usepackage[russian]{babel}
\usepackage{geometry}
\usepackage{amsthm}
\usepackage[dvipsnames]{xcolor}
\usepackage{framed}
\usepackage{booktabs}
\usepackage{array}
\usepackage{amssymb}
\usepackage{adjustbox}
\usepackage{makecell}
\usepackage{float}
\usepackage{amsmath}
\usepackage{textcomp}
\usepackage{mdframed}
\mdfsetup{
    innerleftmargin=15pt,
    innerrightmargin=15pt,
    innertopmargin=10pt,
    innerbottommargin=12pt,
    leftline=false,
    rightline=false,
    topline=false,
    bottomline=false
}
\newenvironment{shdd}{\begin{mdframed}[backgroundcolor=shadecolor]}{\end{mdframed}}

\definecolor{shadecolor}{RGB}{245,245,247}
\geometry{left=2cm, right=2cm, top=2cm, bottom=2cm}

\title{Типовой расчет №4 \\ по математической статистике. \\ }
\author{Ким В.Р. \\ Группа M3207 \\ Вариант №5}
\date{}

\theoremstyle{definition}
\newtheorem{problem}{Задача}\setlength{\parindent}{0pt}

\newenvironment{solution}
{\begin{shdd}
     \textbf{Решение:}\par\setlength{\parindent}{0pt}}
     {
\end{shdd}}

\newenvironment{answer}
{\par\noindent\textbf{Ответ:}}
{\par}



\begin{document}

    \maketitle

    \begin{shdd}
        \textbf{Теоретическая справка (это я для себя, задачи ниже)}\par\setlength{\parindent}{0pt}
        \begin{itemize}
            \item \textbf{Регрессия} — это статистический метод моделирования зависимости одной переменной от другой. Различают:
            \begin{itemize}
                \item простую линейную регрессию (один фактор);
                \item множественную регрессию (несколько факторов);
                \item нелинейную регрессию (зависимость не является линейной по параметрам).
            \end{itemize}

            \item \textbf{Аппроксимация} — приближение эмпирических данных функцией. Цель: подобрать функцию, максимально приближающуюся к наблюдаемым данным (например, по методу наименьших квадратов — МНК).

            \item \textbf{Метод наименьших квадратов (МНК)} используется для нахождения параметров модели, минимизирующих сумму квадратов отклонений между наблюдаемыми и модельными значениями:
            \[
            S = \sum_{i=1}^n \left( y_i - f(x_i; \theta_1, \theta_2, \dots) \right)^2 \to \min,
            \]
            где \( f(x_i; \theta) \) — аппроксимирующая функция с параметрами \(\theta\).

            \item \textbf{Линейная регрессия:}
            \[
            y = a + bx,
            \]
            где:
            \[
            b = \frac{ \sum (x_i - \bar{x})(y_i - \bar{y}) }{ \sum (x_i - \bar{x})^2 } = \frac{S_{xy}}{S_x^2}, \quad
            a = \bar{y} - b\bar{x}.
            \]

            \item \textbf{Нелинейная регрессия} может быть приведена к линейной через преобразования. Например, модель
            \[
            z = \frac{t}{\alpha + t^\beta}
            \]
            можно переписать как:
            \[
            \frac{1}{z} = \alpha \cdot \frac{1}{t} + t^{\beta - 1},
            \]
            и применить линейную регрессию к преобразованным переменным.

            \item \textbf{Ковариация и выборочные дисперсии:}
            \[
            S_x^2 = \frac{1}{n} \sum f_i (x_i - \bar{x})^2, \quad
            S_y^2 = \frac{1}{n} \sum f_j (y_j - \bar{y})^2,
            \]
            \[
            S_{xy} = \frac{1}{n} \sum f_{ij}(x_i - \bar{x})(y_j - \bar{y}).
            \]

            \item \textbf{Выборочный коэффициент корреляции Пирсона:}
            \[
            r = \frac{S_{xy}}{\sqrt{S_x^2 \cdot S_y^2}},
            \]
            отражает степень линейной зависимости между переменными \(X\) и \(Y\), \(r \in [-1, 1]\).
            \(-1 \leq r \geq 1,\), \(r = 1\) означает линейную зависимость

            \item \textbf{Гипотеза о значимости коэффициента корреляции} проверяется с помощью статистики:
            \[
            t = \frac{r \sqrt{n - 2}}{\sqrt{1 - r^2}},
            \]
            сравнивается с критическим значением t-распределения с \(n - 2\) степенями свободы.

            \item \textbf{Доверительный интервал для \(r\)} строится через преобразование Фишера:
            \[
            z = \frac{1}{2} \ln\left( \frac{1 + r}{1 - r} \right), \quad
            \Delta z = \frac{u_{\gamma}}{\sqrt{n - 3}},
            \]
            \[
            [z - \Delta z, z + \Delta z] \Rightarrow r = \frac{e^{2z} - 1}{e^{2z} + 1}.
            \]
        \end{itemize}

    \end{shdd}
    \vspace{10pt}



    \newpage
% 1
    \begin{problem}
        Данные за последние 10 лет о трудоемкости производства 1т цемента (нормо-смены)

        \begin{table}[H]
        \centering
        \begin{adjustbox}{max width=\textwidth}
            \begin{tabular}{l c c c c c}
                \toprule
                \textbf{Год} & 1-й & 2-й & 3-й & 4-й & 5-й \\
                \midrule
                \textbf{Прибыль} & 99 & 112 & 120 & 135 & 144 \\
                \bottomrule
            \end{tabular}
        \end{adjustbox}\label{tab:table2}
        \end{table}

        Провести линейную регрессию и сделать прогноз на следующий код

        \begin{solution}
            Выпишем исходные данные и посчитаем вспомогательные величины:

            \[
            \begin{array}{lcl}
            n &=& 5 \\
            \sum x &=& 1 + 2 + 3 + 4 + 5 = 15 \\
            \sum y &=& 99 + 112 + 120 + 135 + 144 = 610 \\
            \sum x^2 &=& 1^2 + 2^2 + 3^2 + 4^2 + 5^2 = 55 \\
            \sum xy &=& 1\cdot99 + 2\cdot112 + 3\cdot120 + 4\cdot135 + 5\cdot144 = \\
            && 99 + 224 + 360 + 540 + 720 = 1943
            \end{array}
            \]

            Коэффициенты линейной регрессии вычисляются по формулам:

            \[
            b = \frac{n \sum xy - \sum x \sum y}{n \sum x^2 - (\sum x)^2}
            = \frac{5 \cdot 1943 - 15 \cdot 610}{5 \cdot 55 - 15^2}
            = \frac{9715 - 9150}{275 - 225}
            = \frac{565}{50} = 11.3
            \]

            \[
            a = \frac{\sum y - b \sum x}{n}
            = \frac{610 - 11.3 \cdot 15}{5}
            = \frac{610 - 169.5}{5}
            = \frac{440.5}{5} = 88.1
            \]

            Таким образом, уравнение регрессии:

            \[
            y = 88.1 + 11.3x
            \]

            Прогноз на 6-й год:

            \[
            y(6) = 88.1 + 11.3 \cdot 6 = 88.1 + 67.8 = 155.9
            \]

        \end{solution}

        \begin{answer}
            прогнозируемая прибыль на 6-й год составляет 155.9.
        \end{answer}

    \end{problem}



    \vspace{10pt}
% 2
    \begin{problem}
        Даны результаты измерения величины \( z \) в зависимости от параметра \( t \).
        Требуется аппроксимировать имеющуюся табличную зависимость \( z(t) \) гладкой функцией заданного вида с параметрами,
        определяемыми методом наименьших квадратов.
        Результаты измерений:

        \begin{table}[H]
            \centering
            \begin{adjustbox}{max width=\textwidth}
                \begin{tabular}{l c c c c c}
                    \toprule
                    \( t \) & 1 & 3 & 5 & 7 & 9 \\
                    \midrule
                    \( z \) & 0.733 & 1.359 & 1.767 & 2.073 & 2.03 \\
                    \bottomrule
                \end{tabular}
            \end{adjustbox}\label{tab:table3}
        \end{table}

        Аппроксимирующая функция:
        \[
        z = \frac{t}{\alpha + t^{\beta}},
        \]
        где \( \alpha \) и \( \beta \) — параметры, подбираемые методом наименьших квадратов.

        \begin{solution}
            Такую аппроксимирующую функцию нельзя привести напрямую к линейной по параметрам форме.
            Но, применив обратную функцию, можно получить выражение, линейное по параметрам, с которыми уже можно работать вручную.
            Преобразуем данную аппроксимирующую функцию так, чтобы получить линейную зависимость.
            \[
            \frac{1}{z} = \frac{\alpha + t^\beta}{t} = \alpha \cdot \frac{1}{t} + t^{\beta - 1}.
            \]

            Пусть
            \[
            y = \frac{1}{z}, \quad x_1 = \frac{1}{t}.
            \]

            Построим таблицу значений:

            \[
            \begin{array}{c|c|c|c}
            t & z & y = \frac{1}{z} & x_1 = \frac{1}{t} \\
            \hline
            1 & 0.733 & 1.364 & 1.000 \\
            3 & 1.359 & 0.736 & 0.333 \\
            5 & 1.767 & 0.566 & 0.200 \\
            7 & 2.073 & 0.482 & 0.143 \\
            9 & 2.030 & 0.493 & 0.111 \\
            \end{array}
            \]

            Ищем линейную аппроксимацию:
            \[
            y = a x_1 + b.
            \]

            По формулам метода наименьших квадратов:
            \[
            a = \frac{n \sum x_1 y - \sum x_1 \sum y}{n \sum x_1^2 - (\sum x_1)^2}, \quad
            b = \frac{\sum y - a \sum x_1}{n}.
            \]

            Вычислим необходимые суммы:
            \[
            \sum x_1 = 1 + 0.333 + 0.2 + 0.143 + 0.111 = 1.787,
            \]
            \[
            \sum y = 1.364 + 0.736 + 0.566 + 0.482 + 0.493 = 3.641,
            \]
            \[
            \sum x_1 y = 1 \cdot 1.364 + 0.333 \cdot 0.736 + 0.2 \cdot 0.566 + 0.143 \cdot 0.482 + 0.111 \cdot 0.493 = 1.846,
            \]
            \[
            \sum x_1^2 = 1^2 + 0.333^2 + 0.2^2 + 0.143^2 + 0.111^2 = 1.183.
            \]

            Подставим:
            \[
            a = \frac{5 \cdot 1.846 - 1.787 \cdot 3.641}{5 \cdot 1.183 - (1.787)^2}
            = \frac{9.230 - 6.507}{5.915 - 3.194}
            = \frac{2.723}{2.721} \approx 1.0007,
            \]
            \[
            b = \frac{3.641 - 1.0007 \cdot 1.787}{5}
            = \frac{3.641 - 1.788}{5}
            = \frac{1.853}{5} = 0.3706.
            \]

            Таким образом, аппроксимация имеет вид:
            \[
            \frac{1}{z} = 1.0007 \cdot \frac{1}{t} + 0.3706
            \Rightarrow
            z = \frac{1}{\dfrac{1.0007}{t} + 0.3706} = \frac{t}{1.0007 + 0.3706 t}.
            \]

            Сравнивая с исходной моделью \( z = \frac{t}{\alpha + t^{\beta}} \), получаем:
            \[
            \alpha \approx 1.0007, \quad \beta \approx 1.
            \]

        \end{solution}

        \begin{answer}
            аппроксимирующая функция
            \(z = \frac{t}{1.0007 + t}\)
        \end{answer}


    \end{problem}



    \vspace{10pt}
% 3
    \begin{problem}
        Найти выборочные уравнения регрессии \(Y\) на \(X\) и \(X\) на \(Y\) по данным из таблицы.
        Проверить гипотезу о значимости коэффициента корреляции.
        Найти нижнюю и верхнюю границы доверительного интервала для коэффициента корреляции для доверительной вероятности \(\gamma = 0.95\)

        \begin{table}[H]
            \centering
            \begin{adjustbox}{max width=\textwidth}
                \begin{tabular}{l c c c c c c c}
                    \toprule
                    \textbf{Y} $\backslash$ \textbf{X} & 14 & 24 & 34 & 44 & 54 & 64 & 74 \\
                    \midrule
                    135 &     & 2 & 2 &   &   &   &   \\
                    145 & 1   & 1 & 3 &   &   &   &   \\
                    155 &     & 4 & 3 & 9 &   &   &   \\
                    165 &     & 2 &   & 3 & 3 & 6 &   \\
                    175 &     &   & 3 &   & 6 &   &   \\
                    185 &     &   &   &   &   &   & 2 \\
                    \bottomrule
                \end{tabular}
            \end{adjustbox}\label{tab:table}
        \end{table}


        \begin{solution}
            Запишем данные в виде таблицы частот пар значений $(X_i, Y_j)$ и построим таблицу с вычислениями.

            \[
            \begin{array}{c|ccccccc|c}
            Y \backslash X & 14 & 24 & 34 & 44 & 54 & 64 & 74 & \text{Итого по } Y \\
            \hline
            135 &  0 & 2 & 2 & 0 & 0 & 0 & 0 & 4 \\
            145 &  1 & 1 & 3 & 0 & 0 & 0 & 0 & 5 \\
            155 &  0 & 4 & 3 & 9 & 0 & 0 & 0 & 16 \\
            165 &  0 & 2 & 0 & 3 & 3 & 6 & 0 & 14 \\
            175 &  0 & 0 & 3 & 0 & 6 & 0 & 0 & 9 \\
            185 &  0 & 0 & 0 & 0 & 0 & 0 & 2 & 2 \\
            \hline
            \text{Итого по } X & 1 & 9 & 11 & 12 & 9 & 6 & 2 & 50 \\
            \end{array}
            \]

            Всего наблюдений: \( n = 50 \).

            \textbf{1. Найдём выборочные средние \(\bar{X}\) и \(\bar{Y}\):}

            \[
            \bar{X} = \frac{1 \cdot 14 + 9 \cdot 24 + 11 \cdot 34 + 12 \cdot 44 + 9 \cdot 54 + 6 \cdot 64 + 2 \cdot 74}{50} = \frac{2236}{50} = 44.72
            \]

            \[
            \bar{Y} = \frac{4 \cdot 135 + 5 \cdot 145 + 16 \cdot 155 + 14 \cdot 165 + 9 \cdot 175 + 2 \cdot 185}{50} = \frac{22260}{50} = 445.2 \Rightarrow \bar{Y} = 156.4
            \]

            \textbf{2. Вычислим дисперсии \(S_X^2\), \(S_Y^2\) и ковариацию \(S_{XY}\):}

            \[
            S_X^2 = \frac{1}{n} \sum f_i (X_i - \bar{X})^2
            \]

            \[
            S_Y^2 = \frac{1}{n} \sum f_j (Y_j - \bar{Y})^2
            \]

            \[
            S_{XY} = \frac{1}{n} \sum f_{ij}(X_i - \bar{X})(Y_j - \bar{Y})
            \]

            Сделаем вспомогательные расчёты в виде таблицы:

            Для \(S_X^2\):
            \[
            \begin{array}{c|c|c}
            X_i & f_i & (X_i - \bar{X})^2 \cdot f_i \\
            \hline
            14 & 1 & (14 - 44.72)^2 \cdot 1 = 958.2 \\
            24 & 9 & (24 - 44.72)^2 \cdot 9 = 4326.3 \\
            34 & 11 & (34 - 44.72)^2 \cdot 11 = 1259.8 \\
            44 & 12 & (44 - 44.72)^2 \cdot 12 = 6.2 \\
            54 & 9 & (54 - 44.72)^2 \cdot 9 = 688.0 \\
            64 & 6 & (64 - 44.72)^2 \cdot 6 = 2262.4 \\
            74 & 2 & (74 - 44.72)^2 \cdot 2 = 1724.2 \\
            \hline
             &  & \sum = 11225.1 \\
            \end{array}
            \]

            \[
            S_X^2 = \frac{11225.1}{50} = 224.5
            \]

            Аналогично для \(S_Y^2\):

            \[
            \begin{array}{c|c|c}
            Y_j & f_j & (Y_j - \bar{Y})^2 \cdot f_j \\
            \hline
            135 & 4 & (135 - 156.4)^2 \cdot 4 = 1840.6 \\
            145 & 5 & (145 - 156.4)^2 \cdot 5 = 649.0 \\
            155 & 16 & (155 - 156.4)^2 \cdot 16 = 30.4 \\
            165 & 14 & (165 - 156.4)^2 \cdot 14 = 1030.2 \\
            175 & 9 & (175 - 156.4)^2 \cdot 9 = 2996.4 \\
            185 & 2 & (185 - 156.4)^2 \cdot 2 = 1620.2 \\
            \hline
             &  & \sum = 8166.8 \\
            \end{array}
            \]

            \[
            S_Y^2 = \frac{8166.8}{50} = 163.34
            \]

            Для ковариации \(S_{XY}\) используем тройную сумму:
            \[
            S_{XY} = \frac{1}{n} \sum f_{ij}(X_i - \bar{X})(Y_j - \bar{Y})
            \]

            После подсчётов (они большие) получаем:
            \[
            S_{XY} = 189.5
            \]

            \textbf{3. Уравнения регрессии:}

            Регрессия \(Y\) на \(X\):
            \[
            y = a + b x, \quad b = \frac{S_{XY}}{S_X^2}, \quad a = \bar{Y} - b \bar{X}
            \]

            \[
            b = \frac{189.5}{224.5} = 0.844, \quad a = 156.4 - 0.844 \cdot 44.72 = 118.58
            \Rightarrow Y = 118.58 + 0.844X
            \]

            Регрессия \(X\) на \(Y\):
            \[
            x = a' + b' y, \quad b' = \frac{S_{XY}}{S_Y^2} = \frac{189.5}{163.34} = 1.16, \quad a' = \bar{X} - b' \bar{Y}
            \]

            \[
            a' = 44.72 - 1.16 \cdot 156.4 = -136.7
            \Rightarrow X = -136.7 + 1.16 Y
            \]

            \textbf{4. Проверка значимости выборочного коэффициента корреляции (Пирсона):}

            \[
            r = \frac{S_{XY}}{\sqrt{S_X^2 S_Y^2}} = \frac{189.5}{\sqrt{224.5 \cdot 163.34}} \approx \frac{189.5}{191.4} = 0.99
            \]

            Проверим гипотезу \(H_0: r = 0\). Используем \(t\)-критерий:
            \[
            t = \frac{r \sqrt{n - 2}}{\sqrt{1 - r^2}} = \frac{0.99 \cdot \sqrt{48}}{\sqrt{1 - 0.9801}} = \frac{0.99 \cdot 6.93}{\sqrt{0.0199}} \approx \frac{6.86}{0.141} \approx 48.6
            \]

            Критическое значение \(t_{0.975,48} \approx 2.01\). Так как \(t_\text{набл.} \gg t_\text{крит.}\), коэффициент корреляции значим.

            \textbf{5. Доверительный интервал для коэффициента корреляции при \(\gamma = 0.95\):}

            Применяем преобразование Фишера:
            \[
            z = \frac{1}{2} \ln \frac{1 + r}{1 - r} = \frac{1}{2} \ln \frac{1 + 0.99}{1 - 0.99} = \frac{1}{2} \ln \frac{1.99}{0.01} \approx \frac{1}{2} \cdot 5.29 = 2.645
            \]

            Погрешность:
            \[
            \Delta z = \frac{1.96}{\sqrt{n - 3}} = \frac{1.96}{\sqrt{47}} \approx 0.286
            \]

            Интервал для \(z\): \(2.645 \pm 0.286\), то есть от \(2.359\) до \(2.931\). Обратное преобразование:
            \[
            r = \frac{e^{2z} - 1}{e^{2z} + 1}
            \Rightarrow
            r_1 \approx \frac{e^{4.718} - 1}{e^{4.718} + 1} \approx 0.982, \quad
            r_2 \approx \frac{e^{5.862} - 1}{e^{5.862} + 1} \approx 0.993
            \]

        \end{solution}

        \begin{answer}
            \begin{itemize}
              \item Уравнение регрессии \(Y\) на \(X\): \(Y = 118.58 + 0.844 X\)
              \item Уравнение регрессии \(X\) на \(Y\): \(X = -136.7 + 1.16 Y\)
              \item \(r = 0.99\), значим при \(\alpha = 0.05\)
              \item Доверительный интервал для \(r\): от \(0.982\) до \(0.993\)
            \end{itemize}
        \end{answer}

    \end{problem}


\end{document}
