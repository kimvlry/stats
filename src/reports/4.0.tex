% chktex-file 44

\documentclass[a4paper,11pt]{article}
\usepackage[utf8]{inputenc}
\usepackage[russian]{babel}
\usepackage{geometry}
\usepackage{amsthm}
\usepackage[dvipsnames]{xcolor}
\usepackage{framed}
\usepackage{booktabs}
\usepackage{array}
\usepackage{amssymb}
\usepackage{adjustbox}
\usepackage{makecell}
\usepackage{float}
\usepackage{amsmath}
\usepackage{textcomp}
\usepackage{mdframed}
\mdfsetup{
    innerleftmargin=15pt,
    innerrightmargin=15pt,
    innertopmargin=10pt,
    innerbottommargin=12pt,
    leftline=false,
    rightline=false,
    topline=false,
    bottomline=false
}
\newenvironment{shdd}{\begin{mdframed}[backgroundcolor=shadecolor]}{\end{mdframed}}

\definecolor{shadecolor}{RGB}{245,245,247}
\geometry{left=2cm, right=2cm, top=2cm, bottom=2cm}

\title{Типовой расчет №4 \\ по математической статистике. \\ }
\author{Ким В.Р. \\ Группа M3207 \\ Вариант №5}
\date{}

\theoremstyle{definition}
\newtheorem{problem}{Задача}\setlength{\parindent}{0pt}

\newenvironment{solution}
{\begin{shdd}
     \textbf{Решение:}\par\setlength{\parindent}{0pt}}
     {
\end{shdd}}

\newenvironment{answer}
{\par\noindent\textbf{Ответ:}}
{\par}



\begin{document}

    \maketitle

    \begin{shdd}
        \textbf{Теоретическая справка}\par\setlength{\parindent}{0pt}
        \begin{itemize}

        \end{itemize}
    \end{shdd}
    \vspace{10pt}



    \newpage
% 1
    \begin{problem}
        \begin{solution}

        \end{solution}

        \begin{answer}
        \end{answer}

    \end{problem}



    \vspace{10pt}
% 2
    \begin{problem}
        \begin{solution}
        \end{solution}

        \begin{answer}
        \end{answer}


    \end{problem}



    \vspace{10pt}
% 3
    \begin{problem}
        \begin{solution}
        \end{solution}

        \begin{answer}
        \end{answer}

    \end{problem}



    \vspace{10pt}
% 4
    \begin{problem}
        \begin{solution}
        \end{solution}

        \begin{answer}
        \end{answer}

    \end{problem}


    \vspace{10pt}
% 5
    \begin{problem}

        \begin{solution}
        \end{solution}

        \begin{answer}
        \end{answer}

    \end{problem}


\end{document}
